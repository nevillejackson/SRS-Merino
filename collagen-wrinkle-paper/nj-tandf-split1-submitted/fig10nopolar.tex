%\documentclass{article}
%\usepackage{graphicx,subfigure}
%\usepackage{caption,rotating}
%\begin{document}

\begin{figure}[!h]
\centering
\captionsetup{width=0.92\textwidth}
 \subfigure[Sheep w479 Wrinkled]{
%    \includegraphics[scale=0.20]{w479-2-rigid.jpg}
%   \includegraphics[scale=0.10]{fig10a.jpg}
    \includegraphics[width=0.45\textwidth]{fig10a.jpg}
  }
 \subfigure[Sheep w490 Wrinkle-free]{
%    \includegraphics[scale=0.20]{w490-2-supple.jpg}
%   \includegraphics[scale=0.10]{fig10b.jpg}
    \includegraphics[width=0.45\textwidth]{fig10b.jpg}
  }
  \caption{Vertical sections from a wrinkled (a) and a wrinkle-free (b) sheep from Trial 1 flock 2 stained with PSR and examined with a 4x objective. These are the same two sections as shown with polarised light in Figure~\ref{fig:polar}. Skin layers are: {\bf EPID} epidermis, {\bf PD} papillary dermis, and {\bf RD} reticular dermis. Scale bar is $200\mu m$. }
\vfill
  \label{fig:nopolar}
\end{figure}

%\end{document}

