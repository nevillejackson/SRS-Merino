%
% Draft  document histcollwrin.tex
% Looks at how collagen in Merino sheep skin relates to skin wrinkle and follicle curvature 
%
 
\documentclass[titlepage]{article}  % Latex2e
\usepackage{graphicx,lscape,subfigure}
\usepackage{tikz}
\usepackage{bm,longtable}
\usepackage{textcomp}
 

\title{Histology of collagen in Merino sheep skin and its association with skin wrinkle formation and follicle curvature }
\author{The late J.E. Watts S. Maleki, P.G. Swan, J. Gordon, and N. Jackson}
\date{10 June 2019} 

 
\begin{document} 


 
\maketitle      

$\newcommand{\E}{\mathrm{E}}$
$\newcommand{\Var}{\mathrm{Var}}$
$\newcommand{\Cov}{\mathrm{Cov}}$ 
$\newcommand{\SD}{\mathrm{SD}}$ 

\section{Introduction} 

Wrinkle formation in Australian Merino sheep skin is a phenomenon with serious economic and political consequences.  It has long been known (Seddon, Belschner, and Mulhearn (1931)~\cite{sedd:31}) that wrinkled sheep are more susceptible to blowfly strike. The use of the {\em mulesing} operation to control flystrike in Merino sheep has recently been the subject of intense animal ethics scrutiny. No effective alternative management option has appeared.
The most effective long term solution would seem to be to breed the wrinkle out of Merino sheep. This approach has sometimes met with resistance from some Australian Merino breeders who feel that the extra skin surface area of wrinkled sheep is necessary to achieve high levels of wool production. Breeding plans which include some culling on wrinkle usually do not lead to its complete elimination (for example Turner Dolling and Kennedy (1968)~\cite{turn:68}).

This study  is an attempt to go back to the basic biology of wrinkle formation, to see whether we can understand the tissue structure of a wrinkle, and to see if that suggests  a better approach breeding of wrinkle-free sheep,  without lowering productivity or adversely affecting wool quality. 

There have been very few attempts to define what a wrinkle actually is. The early work of Carter(1943)~\cite{cart:43} went as far as describing and naming all the folds on the neck, body, and breech, and developed a set of photographic scores for degree of wrinkle. Carter used the terms {\em fold} and {\em wrinkle} interchangeably, but he distinguished the small {\em pin wrinkles} present in all Merinos, from the larger folds which develop to varying degrees as the sheep matures. From this early start, there is, somewhat surprisingly, nothing on the biology of wrinkles, until the study of Mitchell et al(1984)~\cite{mitc:84}. 

The Mitchell et al(1984)~\cite{mitc:84} paper defines five tissue layers in sheep skin.
\begin{description}
\item[Layer1] epidermis is mainly keratinised protein
\item[Layer2] contains wool follicles and accessory glands, and is part of the dermis. Sometimes called {\em papillary layer}.
\item[Layer3] layers 2 and 3 together called 'dermis' . Contains fibrous proteins, collagen, and elastin. Sometimes called {\em reticular layer} although the structure is not always reticular, but may be interwoven.
\item[Layer4] contains voluntary muscle, collagen and elastin
\item[Layer5] adipose tissue
\end{description}
 These are illustrated in Figure ~\ref{fig:mitchell}
%\documentclass{article}
%\usepackage{graphicx,subfigure}
%\begin{document}

\begin{figure}[!h]
  \centering
  \includegraphics[width=1.0\textwidth]{sanazcollagenwrinkle-img1.jpg}
  \caption{Merino sheep skin showing layers. 1. epidermis with wool fibres; 2. papillary layer of dermis; 3. reticular layer of dermis; 4. areolar tissue and muscle; and 5. adipose tissue. Two wrinkles are present; one alongside each side of the forceps (from Mitchell et al (1984)~\cite{mitc:84})}
  \label{fig:mitchell}
\end{figure}

%\end{document}


Only the first 3 layers curve upward in a folded section of skin, layers 4 and 5 remain straight. This can be seen in Figure ~\ref{fig:mitchell}. Mitchell et al note that Layer2 is much weaker than Layer 3 ( collagen not as hard). When wrinkles or folds occur in the skin, Layers 1,2, and 3 buckle up into a fold, while Layers 4-5 are straight. It appears as if wrinkles are formed either by an overgrowth of Layers 1-3, or by a shrinkage or tightening of Layer 4. Mitchell has demonstrated this by showing that if Layer4 (and Layer 5) are dissected away from a skin specimen with wrinkles, the folds in Layers 1-3 flatten out. So in a wrinkled sheep, Layer 4 is holding the skin under some tension, which relaxes when Layer 4 is removed.

Even less is known about wrinkle development. Merino lambs are born with visible wrinkles.  A somewhat obsure reference (Bogolyubsky (1940)~\cite{bogo:40}) asserts that wrinkles were observed forming in foetal skin of Karakul and Merino lambs at around 100 days of gestation. That is about the time at which the secondary derived follicles initiate. Carter(1943)~\cite{cart:43} presents a photograph of the skin surface of a 10 day old Merino lamb (Plate 13 Figure 1) which clearly shows small {\em pin wrinkles}. There are no other studies of foetal wrinkle development, but there is a considerable literature on follicle development ( see Fraser and Short(1960)~\cite{fras:60} and Maddocks and Jackson(1988)~\cite{madd:88} and Ryder and Stevenson(1968)~\cite{ryde:68} for reviews). There is some literature on collagen development in sheep skin, and we will look at that below.

What is to be investigated in this study is that the amount and type ( and maybe timing and arrangement in the skin) of collagen development might be a factor involved with both wrinkle development and follicle development. So what is known about collagen? Well, it is already present in the dermis (layers 2 and 3) of foetal skin at the time follicles develop (Knight et al (1993) ~\cite{knig:93}). 
 These authors distinguish two collagen types ( Type III or 'soft' collagen, and Type I or 'hard' collagen) and note  that Type III is highest at 75 days of gestation, and falls progressively as the foetus develops, while Type I is low at day 75 and rises to over 50 percent by birth. Collagen fibres are formed from cells called {\em fibroblasts}. At 75-80 days the fibroblasts appear as plump, immature cells surrounded by reticular collagen fibres which are composed of Type III collagen. By birth the fibroblasts have matured  and the collagen fibres may be intermeshed to varying degrees. If the fine reticular fibre pattern remains, it is soft collagen, if the fibres intermesh the collagen tissue is hardened to various degrees. 

Collagen development, secondary follicle development and wrinkle formation  all seem to commence at the same time of around 100 days of foetal age.  Follicle development ceases at around birth ( 150 days) but development of collagen and wrinkles continues into the adult sheep. In this study we look at the end points of development - that is we study collagen and follicles in adult sheep with and without wrinkles. That will not reveal the details of development, but it should make clear any obvious associations between collagen, wrinkles, and follicles.

\section{Materials and Methods}
\subsection{Sheep studied}

\subsection{Measurements}

\subsection{Interpretation of measurements}


\subsection{Statistical Methods}
Data were imported into the R statistical program~\cite{rprog:13} and analysed using the {\em lm()} function for regressions, and the {\em aov()} function for analysis of variance.

\section{Results}


\clearpage
\section{Discussion}


\clearpage
\begin{thebibliography}{99}

\bibitem{bogo:40}
 Bogolyubsky S.N. (1940) cited by Fraser A.S and Short B.F. (1960) The Biology of the Fleece. Animal Research Laboratories Technical Paper No 3. CSIRO Melbourne 1960.

\bibitem{brow:68}
Brown, G.H., and Turner, Helen Newton. (1968) Response to selection in Australian Merino sheep. II. Estimates of phenotypic and genetic parameters for some production traits in Merino ewes and an analysis of the possible effects of selection on them. Aust. J. Agric. Res. 19:303-22

\bibitem{cart:43}
Carter H.B. (1943) Studies in the biology of the skin and fleece of sheep. 1. The development and general histology of the follicle group in the skin of the Merino. 2. The use of tanned sheepskin in the study of follicle population density. 3. Notes on the arrangement, nomenclature, and variation of skin folds and wrinkles in the Merino. C.S.I.R. Bulletin No 164, Melbourne, 1943

\bibitem{fras:60}
Fraser A.S and Short B.F. (1960) The Biology of the Fleece. Animal Research Laboratories Technical Paper No 3. CSIRO Melbourne 1960.

\bibitem{gord:08}
Gordon-Thompson, C., Botto, S.A., Cam, G.R., and Moore, G.P.H. (2008) Notch pathway gene expression and wool follicle cell fates. Aust. J. Exp. Agric. 48(5) 648-656

\bibitem{jack:75}
Jackson, N., Nay, T, and Turner, Helen Newton (1975) Response to selection in Australian Merino sheep. VII Phenotypic and genetic parameters for some wool follicle characteristics and their correlation with wool and body traits. Aust. J. Agric. Res. 26:937-57

\bibitem{jack:15}
Jackson, N. (2015) Genetic relationship betweeen skin and wool traits in Merino sheep. Incomplete manuscript.

\bibitem{jack:17}
Jackson, N. (2017) Genetics of primary and secondary fibre diameters and densities in Merino sheep. URL https://github.com/nevillejackson/atavistic-sheep/mev-rewrite/supplementary/genetic-parameters/psparam.pdf

\bibitem{jack:17a}
Jackson, N. (2017) Genetic relationship between skin and wool traits in Merino sheep. Part I Responses to selection ans estimates of genetic parameters. URL https://github.com/nevillejackson/Fleece-genetics/tree/master/skinandfleeceparameters/ab3220/skinwool1.pdf

\bibitem{jack:18}
Jackson, N. and Watts, J.E. (2018) Does follicle development affect the spatial layout of sheep skin? URL https://github.com/nevillejackson/Fleece-biology/tree/master/skinspace/skinspace.pdf

\bibitem{jack:90}
Jackson, N., Maddocks, I.G., Lax, J., Moore, G.P.M. and Watts, J.E. (1990) Merino Evolution, Skin Characteristics, and Fleece Quality. URL https://github.com/nevillejackson/atavistic-sheep/mev/evol.pdf 

\bibitem{jack:17b}
Jackson, N. and Watts, J.E. (2017) What is known about the genetics of wrinkle score in Merino sheep? URL https://github.com/nevillejackson/Fleece-genetics/wrinkle/wrinkle.pdf

\bibitem{knig:93}
Knight, K.R., Lepore, D.A., Horne, R.S., Ritz, M., Kumta, S. and O'Brian, B.M. (1993) Collagen content of uninjured skin and scar tissue in foetal and adult sheep. Int. J. Exp. Pathol. 74(6):583-591

\bibitem{madd:88}
Maddocks, I.G. and Jackson, N. (1988) Structural studies of sheep, cattle, and goat skin. CSIRO, Division of Aimal Production, Sydney.

\bibitem{ment:80}
Menton, D.N. and Hess, R.A. (1980) The ultrastructure of collagen in the dermis of tight-skin (Tsk) mutant mice. The Journal of Investigative Dermatology 74:139-147

\bibitem{mitc:84}
Mitchell, T.W. et al (1984) Some physical and mechanical properties of sheep akin with a comparison of "thick" and "thin" skins. Wool Technology and Sheep Breeding, Vol XXXII, No IV, 200-206

\bibitem{moor:89}
Moore G.P.M., Jackson, N., and Lax, J. (1989) Evidence of a unique developmental mechanism specifying both wool follicle density and fibre size in sheep selected for single skin and fleece characters. Genet. Res. Camb. 53:57-62

\bibitem{moor:98}
Moore, G.P.M., Jackson, N., Isaacs, K., and Brown, G (1998) J. Theoretical Biology 191:87-94

\bibitem{nay:66}
Nay, T. (1966) Wool follicle arrangement and vascular pattern in the Australian Merino. Aust. J. Agric. Res. 17:797-805

\bibitem{rprog:13}
R Core Team (2013). R: A language and environment for statistical
  computing. R Foundation for Statistical Computing, Vienna, Austria.
  ISBN 3-900051-07-0, URL http://www.R-project.org/.

\bibitem{ryde:68}
Ryder, M.L. and Stevenson, S.K.(1968) Wool Growth. Academic Press, London.

\bibitem{sedd:31}
Seddon, H.R., Belschner, H.G. and Mulhearn, C.R. (1931)  Studies on cutaneous myiasis of sheep. Sew South Walse Department of Agriculture, Science Bulletin No 37, 1931

\bibitem{turn:56} 
Turner, Helen Newton (1956) Anim. Breed. Abstr. 24:87-118

\bibitem{turn:58}
Turner, Helen Newton(1958) Aust. J. Agric. Res. 9:521-52

\bibitem{turn:53}
Turner, Helen Newton, Hayman, R.H., Riches, J.H., Roberts, N.F., and Wilson, L.T. (1953) Physical definition of sheep and their fleece for breeding and husbandry studies: with particular reference to Merino sheep. CSIRO Div. Anim. Hlth. Prod. Div. Rept. No. 4 (Ser SW-2 mimeo)

\bibitem{turn:68}
Turner, H.N., Dolling, C.H.S., and Kennedy, J.F. (1968) Response to selection in Australian Merino sheep. I. Selection for high clean wool weight with a ceiling on fibre diameter and degree of wrinkle. Response in wool and body characteristics. Aust. J. agric. Res. 19:79-112

\bibitem{turn:70}
Turner, Helen Newton, Brooker M.G. and Dolling, C.H.S (1970) Response to selection in Australian Merino sheep. III Single character selection for high and low values of wool weight and its components. Aust.J.Agric.Res. 21:955-84

\bibitem{watt:17}
Watts, J.E., Jackson, N., and Ferguson, K.A. (2017) Improvements in fleece weight weight and wool quality of Merino sheep selected visually for high fibre density and length. URL https://github.com/nevillejackson/SRS-Merino/Paper\_2\_Revised\_10\_November\_2017.docx 

\bibitem{xavi:03}
Xavier, S.P., Gordon-Thomson, C. Wynn, P.C., McCullagh, P., Thomson, P.C., Tomkins, L., Mason, R.S., and Moore, G.P.M.(2003) Evidence that Notch and Delta expressions have a role in dermal condensate aggregation during wool follicle initiation. Experimental Dermatology, 22:656-681

\end{thebibliography}
\end{document}
