%
% Draft  document histcollwrin.tex
% Looks at how collagen in Merino sheep skin relates to skin wrinkle and follicle curvature 
%
 
\documentclass[titlepage]{article}  % Latex2e
\usepackage{graphicx,lscape,subfigure}
\usepackage{caption,rotating}
\usepackage{tikz}
\usepackage{bm,longtable}
\usepackage{textcomp}


\title{Histology of collagen in Merino sheep skin and its association with skin wrinkle formation and follicle curvature }
\author{The late J.E. Watts S. Maleki, P.G. Swan, J. Gordon, and N. Jackson}
\date{10 June 2019} 

 
\begin{document} 


 
\maketitle      

$\newcommand{\E}{\mathrm{E}}$
$\newcommand{\Var}{\mathrm{Var}}$
$\newcommand{\Cov}{\mathrm{Cov}}$ 
$\newcommand{\SD}{\mathrm{SD}}$ 

\section{Introduction} 

Wrinkle formation in Australian Merino sheep skin is a phenomenon with serious economic and political consequences.  It has long been known (Seddon, Belschner, and Mulhearn (1931)~\cite{sedd:31}) that wrinkled sheep are more susceptible to blowfly strike. The use of the {\em mulesing} operation to control flystrike in Merino sheep has recently been the subject of intense animal ethics scrutiny. No effective alternative management option has appeared.
The most effective long term solution would seem to be to breed the wrinkle out of Merino sheep. This approach has at times met with resistance from some Australian Merino breeders who feel that the extra skin surface area of wrinkled sheep is necessary to achieve high levels of wool production. Breeding plans which include some culling on wrinkle usually do not lead to its complete elimination (for example Turner Dolling and Kennedy (1968)~\cite{turn:68}).

This study  is an attempt to go back to the basic biology of wrinkle formation, to see whether we can understand the tissue structure of a wrinkle, and to see if that suggests  a better approach breeding of wrinkle-free sheep,  without lowering productivity or adversely affecting wool quality. 

There have been very few attempts to define what a wrinkle actually is. The early work of Carter(1943)~\cite{cart:43} went as far as describing and naming all the folds on the neck, body, and breech, and developed a set of photographic scores for degree of wrinkle. Carter used the terms {\em fold} and {\em wrinkle} interchangeably, but he distinguished the small {\em pin wrinkles} present in all Merinos, from the larger folds which develop to varying degrees as the sheep matures. From this early start, there is, somewhat surprisingly, nothing on the biology of wrinkles, until the study of Mitchell et al(1984)~\cite{mitc:84}. 

The Mitchell et al(1984)~\cite{mitc:84} paper defines five tissue layers in sheep skin.
\begin{description}
\item[Layer1] epidermis is mainly keratinised protein
\item[Layer2] contains wool follicles and accessory glands, and is part of the dermis. Sometimes called {\em papillary layer}.
\item[Layer3] layers 2 and 3 together called 'dermis' . Contains fibrous proteins, collagen, and elastin. Sometimes called {\em reticular layer} although the structure is not always reticular, but may be interwoven.
\item[Layer4] contains voluntary muscle, collagen and elastin
\item[Layer5] adipose tissue
\end{description}
 These are illustrated in Figure ~\ref{fig:mitchell}
%\documentclass{article}
%\usepackage{graphicx,subfigure}
%\begin{document}

\begin{figure}[!h]
  \centering
  \includegraphics[width=1.0\textwidth]{sanazcollagenwrinkle-img1.jpg}
  \caption{Merino sheep skin showing layers. 1. epidermis with wool fibres; 2. papillary layer of dermis; 3. reticular layer of dermis; 4. areolar tissue and muscle; and 5. adipose tissue. Two wrinkles are present; one alongside each side of the forceps (from Mitchell et al (1984)~\cite{mitc:84})}
  \label{fig:mitchell}
\end{figure}

%\end{document}


Only the first 3 layers curve upward in a folded section of skin, layers 4 and 5 remain straight. This can be seen in Figure ~\ref{fig:mitchell}. Mitchell et al note that Layer2 is much weaker than Layer 3 ( collagen not as hard). When wrinkles or folds occur in the skin, Layers 1,2, and 3 buckle up into a fold, while Layers 4-5 are straight. It appears as if wrinkles are formed either by an overgrowth of Layers 1-3, or by a shrinkage or tightening of Layer 4. Mitchell has demonstrated this by showing that if Layer4 (and Layer 5) are dissected away from a skin specimen with wrinkles, the folds in Layers 1-3 flatten out. So in a wrinkled sheep, Layer 4 is holding the skin under some tension, which relaxes when Layer 4 is removed.

Even less is known about wrinkle development. Merino lambs are born with visible wrinkles.  A somewhat obsure reference (Bogolyubsky (1940)~\cite{bogo:40}) asserts that wrinkles were observed forming in foetal skin of Karakul and Merino lambs at around 100 days of gestation. That is about the time at which the secondary derived follicles initiate. Carter(1943)~\cite{cart:43} presents a photograph of the skin surface of a 10 day old Merino lamb (Plate 13 Figure 1) which clearly shows small {\em pin wrinkles}. There are no other studies of foetal wrinkle development, but there is a considerable literature on follicle development ( see Fraser and Short(1960)~\cite{fras:60} and Maddocks and Jackson(1988)~\cite{madd:88} and Ryder and Stevenson(1968)~\cite{ryde:68} for reviews). There is some literature on collagen development in sheep skin, and we will look at that below.

What is to be investigated in this study is that the amount and type ( and maybe timing and arrangement in the skin) of collagen development might be a factor involved with both wrinkle development and follicle development. So what is known about collagen? Well, it is already present in the dermis (layers 2 and 3) of foetal skin at the time follicles develop (Knight et al (1993) ~\cite{knig:93}). 
 These authors distinguish two collagen types ( Type III or 'soft' collagen, and Type I or 'hard' collagen) and note  that Type III is highest at 75 days of gestation, and falls progressively as the foetus develops, while Type I is low at day 75 and rises to over 50 percent by birth. Collagen fibres are formed from cells called {\em fibroblasts}. At 75-80 days the fibroblasts appear as plump, immature cells surrounded by reticular collagen fibres which are composed of Type III collagen. By birth the fibroblasts have matured  and the collagen fibres may be intermeshed to varying degrees. If the fine reticular fibre pattern remains, it is soft collagen, if the fibres intermesh the collagen tissue is hardened to various degrees. 

Collagen development, secondary follicle development and wrinkle formation  all seem to commence at the same time of around 100 days of foetal age.  Follicle development ceases at around birth ( 150 days) but development of collagen and wrinkles continues into the adult sheep. In this study we look at the end points of development - that is we study collagen and follicles in adult sheep with and without wrinkles. That will not reveal the details of development, but it should make clear any obvious associations between collagen, wrinkles, and follicles.

\section{Materials and Methods}
The experimental design was to choose, by visual inspection, individual sheep with wrinkle-free skin and wrinkly skin from each of a number of Australian Merino flocks. The flocks available for this study were mostly flocks which were undergoing breeding towards the SRS \textsuperscript{TM} Merino type. Consequently most of the sheep chosen as examples of wrinkle-free sheep would have the loose and supple skin which is characteristic of SRS \textsuperscript{TM} Merinos. There is another sort of wrinkle-free sheep which has low follicle density and tight skin and this type is probably not well represented in the present study.

Two trials were conducted
\begin{description}
\item[Trial 1]  Two sheep were chosen from each of five Merino flocks, one wrinkle-free and one wrinkles. This is a randomized block design without replication .  The blocks are the five flocks, and the treatment is the presence or absence of wrinkle.
\item[Trial 2]  Eighteen sheep were chosen from each of two flocks, nine wrinkle-free and nine with wrinkles. This is a randomised block design with replication. The second of these two flocks was more wrinkled and was not breeding towards the SRS \textsuperscript{TM} Merino type.
\end{description} 
\subsection{Skin samples}
In Trial 1 a biopsy sample was taken from the midside position on each sheep and the specimens were trimmed in the normal manner before processing, so that only Layer 1 (epidermis) and Layer 2 (papillary dermis) were present for histological observation.

In Trial 2 , for the sheep with wrinkly skins, skin biopsies were collected from on the wrinkles as well as between the wrinkles. For the wrinkle-free sheep only one biopsy sample was collected. These specimens included Layers 1 to 4, ie only the adipose tissue was trimmed.

Midside skin samples were collected using a 10 millimetre
circular trephine (Acu Punch® skin biopsy punches, Acuderm, Inc.) and
fixed in 10\% formol saline solution. 

\subsection{Macroscopic skin observations}

Skin samples were washed in several changes of water, the wool
stubble trimmed and then examined under a magnifying lamp ( x 3
magnification).  Scores for  suppleness (1 = hardened to 5 = supple)
of the papillary layer and reticular layer  were made.  Each skin
sample was examined to determine if layers 2 and 3, and layers 3 and 4,
were free or fixed and whether localized hardening and folding of the
skin had occurred.

The thicknesses of the papillary dermis and
the reticular dermis were measured using a ruler graduated in one
millimetre divisions. A Mitutoyo ballpoint gauge
(model no. 2046S) was then used to measure the compressed thickness at
four sites for each skin sample.

\subsection{Histological skin processing and observations}
\subsubsection{Collagen observations}

Skin samples used for haematoxylin and eosin staining (H-E) and
picrosirius red (PSR),were fixed in 10\% neutral buffered formalin for
24 hours before being processed to wax in an automated tissue
processing platform (Shandon Excelsior, Thermo Scientific, USA), and
then embedded in paraffin wax. Four micron sections were cut and placed
onto slides for H-E staining for tissue morphology. Serial section was
also employed on a separate slide for PSR staining to highlight
collagen content. Staining was performed manually.

Sections were then reviewed microscopically (BX53 Olympus, Australia)),
and images taken on 3 CCD camera (DP72, Olympus, Australia) under both
bright field and polarized conditions for PSR staining.

For PSR collagen analysis, the 40x objective was employed at a fixed
exposure to take high power images of 5 random deep dermal fields of
view for computational analysis. 

\begin{verbatim}
 [ Sanaz it seems to me that these 5 random fields would have been
 chosen within the red stained areas with collagen present. 
 I think we should say so]
\end{verbatim}

The images for each sample were then uploaded for quantitative analysis
via the ImagePro Plus (Media Cybernetics, USA) 7.1 software in which
thresholds were set to count all pixels comprising of the red staining
fibres in the PSR stained specimen against the total pixels. A mean was
calculated for each of the specimens{\textquoteright} 5 images and
graphed.

Polarised light was employed in order to try and determine the type of
collagen present within each of the samples.
\begin{verbatim}
[Sanaz, you made a comment about this on Jim's last draft.

The yellow and green reflectances are likely to indicate soft (Type III)
 collagen (Sanaz, please check this statement).
 Need to be careful here Jim, as no one has been able to definitively 
 prove the birefringences of PSR staining with collagen fibres,
 and some of the literature contradicts itself.
 I can pull a few papers to reference as a guide to the reviewers?

and this

Birefringence measurements of PSR stained skin sections indicate that
 nearly all  (…. %) of the collagen sheets in the subfollicular
 layer of the papillary dermis have the deep red light reflectance 
indicative of hard (type I) collagen. (Sanaz, please check this statement).
 Again Jim, we have to tread carefully here making definitive
 statements based on colour birefringence. We can certainly point
 out that the thicker fibres were red, and the thinner fibres more
 green, with some yellowish-orange colours in between.

Can you make some statement that is either definitive or indicative
 and give a reference please?

I think it belongs here in the Methods, not mixed up with Results where Jim had it.
\end{verbatim}

\subsubsection{Vertical skin sections}
Vertical skin sections, approximately 0.3 millimetres wide, were cut
freehand with a sharp razor blade on a freezing stage and stained with
0.25 \% Nile blue sulphate, as described by Nay (1973).
The sections were cut parallel with the angle of
emergence of the fibres to avoid cutting through follicles. Mean
follicle curvature was scored from 1 = straight follicles to 7 =
tangled follicles by reference to a set of standard drawings used by
Nay and Johnson (1973). Follicle depth was measured
as both the perpendicular and angular distances (in millimetres)
between the skin surface and the lower ends of the follicle bulbs,
along with follicle bending, as described by Maddocks and Jackson
(1988).


\subsubsection{Horizontal skin sections}
Horizontal skin sections were also prepared as described by
Maddocks and Jackson (1988) using the frozen section technique and
measurement procedures of Nay (1973). The sections were used to measure
follicle density, secondary follicle to primary follicle ratio (S/P
ratio), primary fibre diameter and secondary fibre diameter of the
sheep.

JW to describe measurement of orientation of follicle groups
and measurements made of collagen sheets in subfollicular layer of
papillary dermis.


\subsection{Summary of measurements}


\subsection{Statistical Methods}

Data were imported into the R statistical program~\cite{rprog:13} and analysed using the {\em aov()} function for analysis of variance.

\section{Results}
We follow the path of looking first at overall morphology of skin specimens, then at the details of collagen structure, and finally at other related measurements

\subsection{Macro observations on biopsy specimen}
In Table~\ref{tab:macro} we present the suppleness scores and percent compressibility of specimens from the sheep from Trial 1.
%\documentclass{article}
%\usepackage{lscape}
%\begin{document}

\begin{table}[htp]
\centering
\caption{Suppleness scores and compressibility measurements for Flocks 1 to 5 of Trial 1}
\label{tab:macro}
\vspace{0.1in}
\begin{tabular}{|p{0.6in}|p{0.6in}|p{0.8in}|p{0.8in}|p{0.9in}|}  \hline
     Flock No. & Sheep No.  &  Skin Type & Suppleness Score & Compressibility Percent \\ 
\hline
  1 & W206 & Wrinkle-free & 5 & 75 \\
  1 & W205 & Wrinkled     & 2 & 54 \\
  2 & W490 & Wrinkle-free & 5 & 64 \\
  2 & W479 & Wrinkled     & 2 & 39 \\
  3 & W555 & Wrinkle-free & 5 & 67 \\
  3 & W547 & Wrinkled     & 1 & 58 \\
  4 & W567 & Wrinkle-free & 5 & 70 \\
  4 & W558 & Wrinkled     & 2 & 63 \\
  5 & W283 & Wrinkle-free & 5 & 69 \\
  5 & W290 & Wrinkled     & 2 & 44 \\ \hline
\end{tabular}
\end{table}

%\end{document}

We see that the wrinkled sheep specimens were consistently less supple and less compressible than those of the wrinkle-free sheep.
These differences in Suppleness and Compressibility were tested for significance in an analysis of variance shown in Table~\ref{tab:macrot1aov}
% latex table generated in R 3.4.2 by xtable 1.8-2 package
% Fri Jul  5 17:48:54 2019
\begin{table}[ht]
\centering
\caption{Analysis of variance of Suppleness score and Compressibility}
\label{tab:macrot1aov}
\vspace{0.1in}

\begin{tabular}{lrrrrr}
 & Response Suppleness & & & \\
  \hline
 & Df & Sum Sq & Mean Sq & F value & Pr($>$F) \\ 
  \hline
FlockNo     & 4 & 0.40 & 0.10 & 1.00 & 0.5000 \\
  SkinType    & 1 & 25.60 & 25.60 & 256.00 & 0.0001 \\
  Residuals   & 4 & 0.40 & 0.10 &  &  \\ 
   \hline
\\
 & Response Compressibility & & & \\
  \hline
 & Df & Sum Sq & Mean Sq & F value & Pr($>$F) \\ 
  \hline
FlockNo     & 4 & 305.60 & 76.40 & 1.99 & 0.2608 \\
  SkinType    & 1 & 756.90 & 756.90 & 19.71 & 0.0113 \\
  Residuals   & 4 & 153.60 & 38.40 &  &  \\
   \hline
\end{tabular}
\end{table}


The differences between skin types ( wrinkled and wrinkle-free) were significaant for both Suppleness and Compressibility. Flock differences were not significant.

\subsection{Skin tissue Morphology}
The pairs of wrinkle free and wrinkled sheep from each flock in Trial 1 showed consistent visual differences in their tissue structure. Figure~\ref{fig:trial1he} shows vertical sections stained with H-E from the wrinkled and wrinkle-free pair of sheep from flock 2.

%\documentclass{article}
%\usepackage{graphicx,subfigure}
%\usepackage{caption,rotating}
%\begin{document}

\begin{figure}[p]
\centering
 \subfigure[Plate (i) Sheep w479-2 Wrinkled]{
    \label{fig:trial1he(i)}
    \includegraphics[scale=0.20]{w479-2-rigid.jpg}
% \includegraphics[width=1.0\textwidth]{w479-2-rigid.jpg}
  }
 \subfigure[Plate (ii) Sheep x490-2 Wrinkle-free]{
    \label{fig:trial1he(ii)}
    \includegraphics[scale=0.20]{w490-2-supple.jpg}
  }
  \caption{Vertical sections from a wrinkled (i) and a wrinkle-free (ii)  sheep from Trial 1 flock 2 stained with H-E. }
\vfill
  \label{fig:trial1he}
\end{figure}

%\end{document}



The wrinkled sheep has a considerably greater amount of haematoxylin stained connective tissue in the lower dermis below the deepest follicle bulbs and to some extent in between the deepest bulbs. Also the follicles in the wrinkled sheep are at a variety of angles and are curved, as evidenced by the follicle shafts being sectioned and the follicle bulb being elliptical indicating sectioning at an angle. In contrast the wrinkle-free sheep has only a minimum amount of haematoxylin stained connective tissue below the follicle bulbs, and the sectioned follicles are more uniform. 
These differences were consistent across all 6 flocks in Trial 1.

Because the Trial 1 biopsy samples were trimmed before sectioning, we can not be sure if the apparent extra connective tissue in the wrinkled sheep extends right to the bottom of the lower dermis or into Layer 4.  We can check on this by looking at Trial 2 specimens, which were nt trimmed before sectioning. Figure~\ref{fig:trial2he} shows one example section whch is from a wrinkled sheep from Flock 1 of Trial 2.
%\documentclass{article}
%\usepackage{graphicx,subfigure}
%\begin{document}

\begin{figure}[!h]
  \centering
  \includegraphics[width=1.0\textwidth]{3456_4layers_4x.jpg}
  \caption{Vertical section from a wrinkled sheep (3456) from Trial 2 Flock 1 stained with H-E. This section is from an untrimmed biopsy specimen and shows all 4 layers (Epidermis, Papillary dermis, Reticular dermis and Muscle layer).}
  \label{fig:trial2he}
\end{figure}

%\end{document}



It is clear from Figure~\ref{fig:trial2he} that the connective tissue in Layer 3 (lower or reticular dermis) does not extend further into Layer 4 (muscle layer). The muscle layer has only stained with the pink eosin counterstain and does not show the reticular structure of the connective tissue.

We are therefore concerned with the nature of this connective tissue in the lower dermis only. We wish to quantify and qualify the way in which it differs between wrinkled and wrinkle-free sheep.

\subsection{Detailed morphology of connective tissue} 




\clearpage
\section{Discussion}


\clearpage
\begin{thebibliography}{99}

\bibitem{bogo:40}
 Bogolyubsky S.N. (1940) cited by Fraser A.S and Short B.F. (1960) The Biology of the Fleece. Animal Research Laboratories Technical Paper No 3. CSIRO Melbourne 1960.

\bibitem{brow:68}
Brown, G.H., and Turner, Helen Newton. (1968) Response to selection in Australian Merino sheep. II. Estimates of phenotypic and genetic parameters for some production traits in Merino ewes and an analysis of the possible effects of selection on them. Aust. J. Agric. Res. 19:303-22

\bibitem{cart:43}
Carter H.B. (1943) Studies in the biology of the skin and fleece of sheep. 1. The development and general histology of the follicle group in the skin of the Merino. 2. The use of tanned sheepskin in the study of follicle population density. 3. Notes on the arrangement, nomenclature, and variation of skin folds and wrinkles in the Merino. C.S.I.R. Bulletin No 164, Melbourne, 1943

\bibitem{fras:60}
Fraser A.S and Short B.F. (1960) The Biology of the Fleece. Animal Research Laboratories Technical Paper No 3. CSIRO Melbourne 1960.

\bibitem{gord:08}
Gordon-Thompson, C., Botto, S.A., Cam, G.R., and Moore, G.P.H. (2008) Notch pathway gene expression and wool follicle cell fates. Aust. J. Exp. Agric. 48(5) 648-656

\bibitem{jack:75}
Jackson, N., Nay, T, and Turner, Helen Newton (1975) Response to selection in Australian Merino sheep. VII Phenotypic and genetic parameters for some wool follicle characteristics and their correlation with wool and body traits. Aust. J. Agric. Res. 26:937-57

\bibitem{jack:15}
Jackson, N. (2015) Genetic relationship betweeen skin and wool traits in Merino sheep. Incomplete manuscript.

\bibitem{jack:17}
Jackson, N. (2017) Genetics of primary and secondary fibre diameters and densities in Merino sheep. URL https://github.com/nevillejackson/atavistic-sheep/mev-rewrite/supplementary/genetic-parameters/psparam.pdf

\bibitem{jack:17a}
Jackson, N. (2017) Genetic relationship between skin and wool traits in Merino sheep. Part I Responses to selection ans estimates of genetic parameters. URL https://github.com/nevillejackson/Fleece-genetics/tree/master/skinandfleeceparameters/ab3220/skinwool1.pdf

\bibitem{jack:18}
Jackson, N. and Watts, J.E. (2018) Does follicle development affect the spatial layout of sheep skin? URL https://github.com/nevillejackson/Fleece-biology/tree/master/skinspace/skinspace.pdf

\bibitem{jack:90}
Jackson, N., Maddocks, I.G., Lax, J., Moore, G.P.M. and Watts, J.E. (1990) Merino Evolution, Skin Characteristics, and Fleece Quality. URL https://github.com/nevillejackson/atavistic-sheep/mev/evol.pdf 

\bibitem{jack:17b}
Jackson, N. and Watts, J.E. (2017) What is known about the genetics of wrinkle score in Merino sheep? URL https://github.com/nevillejackson/Fleece-genetics/wrinkle/wrinkle.pdf

\bibitem{knig:93}
Knight, K.R., Lepore, D.A., Horne, R.S., Ritz, M., Kumta, S. and O'Brian, B.M. (1993) Collagen content of uninjured skin and scar tissue in foetal and adult sheep. Int. J. Exp. Pathol. 74(6):583-591

\bibitem{madd:88}
Maddocks, I.G. and Jackson, N. (1988) Structural studies of sheep, cattle, and goat skin. CSIRO, Division of Aimal Production, Sydney.

\bibitem{ment:80}
Menton, D.N. and Hess, R.A. (1980) The ultrastructure of collagen in the dermis of tight-skin (Tsk) mutant mice. The Journal of Investigative Dermatology 74:139-147

\bibitem{mitc:84}
Mitchell, T.W. et al (1984) Some physical and mechanical properties of sheep akin with a comparison of "thick" and "thin" skins. Wool Technology and Sheep Breeding, Vol XXXII, No IV, 200-206

\bibitem{moor:89}
Moore G.P.M., Jackson, N., and Lax, J. (1989) Evidence of a unique developmental mechanism specifying both wool follicle density and fibre size in sheep selected for single skin and fleece characters. Genet. Res. Camb. 53:57-62

\bibitem{moor:98}
Moore, G.P.M., Jackson, N., Isaacs, K., and Brown, G (1998) J. Theoretical Biology 191:87-94

\bibitem{nay:66}
Nay, T. (1966) Wool follicle arrangement and vascular pattern in the Australian Merino. Aust. J. Agric. Res. 17:797-805

\bibitem{rprog:13}
R Core Team (2013). R: A language and environment for statistical
  computing. R Foundation for Statistical Computing, Vienna, Austria.
  ISBN 3-900051-07-0, URL http://www.R-project.org/.

\bibitem{ryde:68}
Ryder, M.L. and Stevenson, S.K.(1968) Wool Growth. Academic Press, London.

\bibitem{sedd:31}
Seddon, H.R., Belschner, H.G. and Mulhearn, C.R. (1931)  Studies on cutaneous myiasis of sheep. Sew South Walse Department of Agriculture, Science Bulletin No 37, 1931

\bibitem{turn:56} 
Turner, Helen Newton (1956) Anim. Breed. Abstr. 24:87-118

\bibitem{turn:58}
Turner, Helen Newton(1958) Aust. J. Agric. Res. 9:521-52

\bibitem{turn:53}
Turner, Helen Newton, Hayman, R.H., Riches, J.H., Roberts, N.F., and Wilson, L.T. (1953) Physical definition of sheep and their fleece for breeding and husbandry studies: with particular reference to Merino sheep. CSIRO Div. Anim. Hlth. Prod. Div. Rept. No. 4 (Ser SW-2 mimeo)

\bibitem{turn:68}
Turner, H.N., Dolling, C.H.S., and Kennedy, J.F. (1968) Response to selection in Australian Merino sheep. I. Selection for high clean wool weight with a ceiling on fibre diameter and degree of wrinkle. Response in wool and body characteristics. Aust. J. agric. Res. 19:79-112

\bibitem{turn:70}
Turner, Helen Newton, Brooker M.G. and Dolling, C.H.S (1970) Response to selection in Australian Merino sheep. III Single character selection for high and low values of wool weight and its components. Aust.J.Agric.Res. 21:955-84

\bibitem{watt:17}
Watts, J.E., Jackson, N., and Ferguson, K.A. (2017) Improvements in fleece weight weight and wool quality of Merino sheep selected visually for high fibre density and length. URL https://github.com/nevillejackson/SRS-Merino/Paper\_2\_Revised\_10\_November\_2017.docx 

\bibitem{xavi:03}
Xavier, S.P., Gordon-Thomson, C. Wynn, P.C., McCullagh, P., Thomson, P.C., Tomkins, L., Mason, R.S., and Moore, G.P.M.(2003) Evidence that Notch and Delta expressions have a role in dermal condensate aggregation during wool follicle initiation. Experimental Dermatology, 22:656-681

\end{thebibliography}
\end{document}
